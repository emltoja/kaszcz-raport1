\documentclass{article}
\usepackage{polski}
\usepackage{tgpagella}
\usepackage{hyperref}
\usepackage[explicit]{titlesec}


\hypersetup{
colorlinks=true,
urlcolor=blue
}

%\titleformat{\section}
%	{\normalfont}{\thesection}{1em}{\MakeUppercase{#1}}

\author{Emil Olszewski, Artur Sadurski}
\date{\today}
\title{Komputerowa analiza szeregów czasowych \\ Raport 1.}


\begin{document}
\maketitle

\begin{abstract}
Przedmiotem analizy jest losowa próbka 5000 danych ze zbioru zaw1ierającego informacje na temat 
męskich trójboistów zrzeszonych w ramach federacji IPF. \href{https://gitlab.com/openpowerlifting/opl-data}{Dane} zostały udostępnione na warunkach licencji GNU AGPLv3. 
Przeanalizowano zależność pomiędzy masą ciała zawodnika a jego wynikiem w kateogrii RAW, . 
\end{abstract}

\section{Aparatura}
Do analizy danych użyto języka \textit{Julia} w wersji \textit{1.9.3} wraz z następującymi bibliotekami:
\begin{itemize}
\item \textit{DataFrames, Statistics} - analiza danych.
\item \textit{Plots} - wykresy i wizualizacja. 
\item \textit{GLM} - model regresji liniowych. 
\end{itemize}


\end{document}